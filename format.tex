%%%%%%%%%%%%%%%%%%%%%%%%%%%%%%%%%%%%%%%%%%%%%%%%%%%%%%%%%%%
% Author: Sébastien Gardoll copyright all rights reserved %
% email: sebastien@gardoll.fr                             %
% LaTeX Project Public License 1.3c                       %
%%%%%%%%%%%%%%%%%%%%%%%%%%%%%%%%%%%%%%%%%%%%%%%%%%%%%%%%%%%

%%%%%%%%%%%%%%%%%%%%%%%%%%%%%%%% DOCUMENT %%%%%%%%%%%%%%%%%%%%%%%%%%%%%%%%%%%%%%

% option draft - debug mode - (faster compilation time).
% option oneside: for web publication.
% option twoside: for paper publication.
% option openright: the chapters begin at the right.
% option final : printer grade.

%%% EXAMPLE OF WEB PUBLICATION %%%
%\documentclass[french,a4paper,12pt,twoside,final]{book}

%%% EXAMPLE OF PAPER PUBLICATION %%%
\documentclass[french,a4paper,12pt,openright,twoside,final]{book}

% See section hyperlink to continue with WEB or PAPER settings.

%%%%%%%%%%%%%%%%%%%%%%%%%%%%%%%%%%%% FONTS %%%%%%%%%%%%%%%%%%%%%%%%%%%%%%%%%%%%%

% XeLateX / LuaLateX font selection package.
\usepackage{fontspec}

% Document font.
\setmainfont{Liberation Serif}

% Localisation.
% french referes to frenchb ; frenchb and francais are deprecated.
% babel fully supportes LuaLateX ; polyglossia doesn't fully support LuaLateX.
\usepackage[french]{babel}

% None breakable spaces for the french quote symboles.
\frenchbsetup{og=«, fg=»}

% Micro-typographical adjustments.
\usepackage{microtype}

%%%%%%%%%%%%%%%%%%%%%%%%%%%%%%%%%%% MATHS %%%%%%%%%%%%%%%%%%%%%%%%%%%%%%%%%%%%%%

% Maths symboles.
\usepackage{amsmath,amsfonts,amssymb,amsthm}
\usepackage{unicode-math} % needed for \setmathfont
% Use this font instead of the default font (latin modern) for math.
\setmathfont{Libertinus Math}
\theoremstyle{plain}
\newtheorem{theo}{Théorème}

%%%%%%%%%%%%%%%%%%%%%%%%%%%%%%%%%% FIGURES %%%%%%%%%%%%%%%%%%%%%%%%%%%%%%%%%%%%%

% Make \section to be a barrier for the floating objects:
% sections define scope for the placement of floating objects.
\usepackage[section]{placeins}

% Sideways of figures & tables.
\usepackage{verbatim}

%%%%%%%%%%%%%%%%%%%%%%%%%%%%%%%%%%% COLORS %%%%%%%%%%%%%%%%%%%%%%%%%%%%%%%%%%%%%

% Define colors.
\usepackage{color}
%\definecolor{mygray}{gray}{0} % 0 for black ; 1 for white.
%\definecolor{linkcol}{rgb}{0,0,0.4}
%\definecolor{citecol}{rgb}{0.5,0,0}

%%%%%%%%%%%%%%%%%%%%%%%%%%%%%%%%%% HYPERLINK %%%%%%%%%%%%%%%%%%%%%%%%%%%%%%%%%%%

%%% Hyper reference configuration %%%

%%% WEB VERSION %%%
\usepackage[colorlinks = true, allcolors = blue]{hyperref}

%%% PAPER VERSION %%%
%\usepackage[colorlinks = true, allcolors = black]{hyperref}

\urlstyle{rm} % Remove the URL special font and use the document font.

%%%%%%%%%%%%%%%%%%%%%%%%%%%%%%%%%% PARAGRAPHS %%%%%%%%%%%%%%%%%%%%%%%%%%%%%%%%%%

% Paragraph indentation.
\setlength{\parindent}{1em}

% Space between paragraphs. Warning, page end will not be consistant.
% If you wish some space between paragraphs, reimplement this template with
% the document class memoir (a lot of work) and add the \nonzeroparskip
% so as to get consistant page ends.
%\setlength{\parskip}{8pt}

% This command modify the outlook the document.
% Using these command leads to modify the spacing of the headers below
% see (\titlespacing) in the section template.
%\renewcommand{\baselinestretch}{1.05} % inter line space factor.

% Manage the widows and orphans by streching the space between lines of the
% faulty paragraph. Warning, this instruction can lead to greater or lower
% paragraph spacing. Call \nowidow[n] at the end of the faulty paragraph,
% with n as the minimal number of lines.
\usepackage{nowidow}

%%%%%%%%%%%%%%%%%%%%%%%%%%%%%%%%%%% MARGINS %%%%%%%%%%%%%%%%%%%%%%%%%%%%%%%%%%%%

%%% Document settings %%%
\usepackage{geometry} % Set a default text scale according to the page.

% Page geometry configuration.
\geometry{left=30mm,right=30mm,top=30mm,bottom=30mm} % Document margins.

%%%%%%%%%%%%%%%%%%%%%%%%%%%%%%%%%%%% LISTS %%%%%%%%%%%%%%%%%%%%%%%%%%%%%%%%%%%%%

%%% Listes %%%

% Cancel french setup for list and enable english style for list (bullet list).
\frenchbsetup{StandardLists=true}

% Enable list fine tuning.
\usepackage{enumitem}

% list configuration keys:
% - leftmargin : bullet margin
% - labelsep   : space after bullet
% - itemsep    : item vertical spacing
%\setlist[itemize]{leftmargin=12pt, labelsep=3pt, itemsep=0pt}
%\setlist[enumerate]{leftmargin=16pt, labelsep=3pt, itemsep=0pt}
\setlist{nosep} % Delete the vectical spacing.

%%%%%%%%%%%%%%%%%%%%%%%%%%%%%%%%% BIBLIOGRAPHY %%%%%%%%%%%%%%%%%%%%%%%%%%%%%%%%%

\usepackage[backend=biber,style=numeric]{biblatex}

%%%%%%%%%%%%%%%%%%%%%%%%%%%%%%%%%%% TABLES %%%%%%%%%%%%%%%%%%%%%%%%%%%%%%%%%%%%%

% Tables on several pages - page anotations.
%\RequirePackage{supertabular}
\RequirePackage{longtable}

% Ctable style.
\usepackage{ctable}

%%%%%%%%%%%%%%%%%%%%%%%%%%%%%%%%%%% OTHERS %%%%%%%%%%%%%%%%%%%%%%%%%%%%%%%%%%%%%

% Glossary / list of abbreviations.
\usepackage[intoc]{nomencl}
\renewcommand{\nomname}{Liste des Abréviations}

% Dummy text package.
\usepackage{lipsum}

% Logos, text superscript, subscript
% and importation of glyph from other fonts.
\ifluatex
  \usepackage{luatextra}
  \newcommand{\engineLogo}{\LuaLaTeX}
\fi

\ifxetex
  \usepackage{xltxtra}
  \newcommand{\engineLogo}{\XeLaTeX}
\fi

% Arithemics with latex.
\usepackage{calc}

%%%%%%%%%%%%%%%%%%%%%%%%%%%%%%%%%%%%% TOC %%%%%%%%%%%%%%%%%%%%%%%%%%%%%%%%%%%%%%

% Table of content for the document and for each chapter.
\usepackage[french]{minitoc}

%%%%%%%%%%%%%%%%%%%%%%%%%%%%%%%% PAGE  TEMPLATE %%%%%%%%%%%%%%%%%%%%%%%%%%%%%%%%

%%%%%%%%%%%%%%%%%%%%%%%%%%  TOC

% [titles] keeps the fancy header style.
\usepackage[titles]{tocloft}

% Configure the spacing in the TOC.
\cftsetindents{section}{0cm}{1cm}
\cftsetindents{subsection}{5pt}{1cm}
\cftsetindents{subsubsection}{10pt}{1cm}

\setcounter{tocdepth}{2} % 2 <=> Display only chapters and sections in the TOC.

%%%%%%%%%%%%%%%%%%%%%%%%%% header and footer

\usepackage{fancyhdr}

% Clear the previous configurations of footer and header.
\fancyhf{}

%\renewcommand{\sectionmark}[1]{\markright{#1}}

\makeatletter
\if@twoside
\makeatother

%%% Header for two sided documents:

% Page number in left on even, right on odd pages.
\fancyfoot[LE,RO]{\thepage}
% Chapter in the right on even pages.
\fancyhead[LE]{\nouppercase{\leftmark}}
% Section in the left on odd pages.
\fancyhead[RO]{\nouppercase{\rightmark}}

\else
%%% Header for one sided documents:

% Page number in the middle of the page.
\fancyfoot[C]{\thepage}
\rhead{\nouppercase{\leftmark}} % Print the only chapter at the right.

\fi

% Define a plain style for pages that are not from a chapter.
\fancypagestyle{plain}
{
  \fancyhead{}
  \renewcommand{\headrulewidth}{0pt} % This remove the header rule.
}

% Clear header style on the last empty odd pages (if the document is twosided).
% This make the first page of chapters to start on even pages.
\makeatletter
\def\cleardoublepage{\clearpage\if@twoside \ifodd\c@page\else%
  \hbox{}%
  % Empty header styles
  \thispagestyle{plain}
  \fancyfoot[LE,RO]{\thepage}
  \newpage%
  \if@twocolumn\hbox{}\newpage\fi\fi\fi}
\makeatother

% Enable the fancy header and foot page.

\pagestyle{fancy}

% Reconfigure the style of the chapter and section number.
% Always add these instructions after \pagestyle{fancy}.
\renewcommand{\sectionmark}[1]{\markright{\thesection~~#1}}
\renewcommand{\chaptermark}[1]{\markboth{\chaptername~\thechapter.~#1}{}}

%%%%%%%%%%%%%%%%%%%%%%%%%%%%%% Chapitre, section, etc.

\setcounter{secnumdepth}{4} % 4 <=> numbered chapter to subsubsection.

% Easy configuration of chapter, section, etc.
\usepackage{titlesec}

\usepackage{etoolbox}

% Solve the titlesec lost of section, subsection, etc. numbering lost.
\makeatletter
\patchcmd{\ttlh@hang}{\parindent\z@}{\parindent\z@\leavevmode}{}{}
\patchcmd{\ttlh@hang}{\noindent}{}{}{}
\makeatother

% Spacing
% \titlespacing{<command>}{<left>}{<before>}{<after>}[<right>]
% Add * so as to delete the indentation.
\titlespacing*{\chapter}{0mm}{50mm}{18mm}[0mm]
\titlespacing*{\section}{0mm}{2mm}{5mm}[0mm]
\titlespacing*{\subsection}{0mm}{2mm}{0mm}[0mm]
\titlespacing*{\subsubsection}{0mm}{2mm}{0mm}[0mm]
\titlespacing*{\paragraph}{0mm}{0mm}{0mm}[0mm]
\titlespacing*{\subparagraph}{0mm}{0mm}{0mm}[0mm]

%%%%%%%%%%% Numbered chapter
\titleformat{\chapter}[display]
{\filleft\huge\bfseries} % Chapter title format
{\makebox[\linewidth][r]{\raisebox{103pt}[0pt][20pt]{\textcolor{gray!50}{\fontsize{80pt}{80pt}\selectfont\thechapter}}}} % Chapter title number.
{-40mm} %sep
{}  %before-code
[\vspace{0.5cm}\titlerule]

%%%%%%%%%%% Unumbered chapter

\titleformat{name=\chapter,numberless}[display]
{\filleft\huge\bfseries} % Chapter title format
{\makebox[10pt][l]{\raisebox{10pt}[0pt][0pt]{\textcolor{gray!50}{\fontsize{10pt}{12pt}\selectfont}}}} % Chapter title number(not displayed).
{-30mm}   %sep
{}  %before-code
[\vspace{0.5cm}\titlerule]

%%%%%%%%%%%% Section, subsection, etc.

% The makebok command so as to align headers.
% The variable headerspacing is tightly related to the format of the headers.
\newlength{\headerspacing}
\setlength{\headerspacing}{40pt}
\titleformat{\section}
  {\normalfont\Large\bfseries}{\makebox[\headerspacing][l]{\thesection}}{0pt}{}

\titleformat{\subsection}
  {\normalfont\large\bfseries}{\makebox[\headerspacing][l]{\thesubsection}}{0pt}{}

\titleformat{\subsubsection}
  {\normalfont\normalsize\bfseries}{\makebox[\headerspacing][l]{\thesubsubsection}}{0pt}{}

% Redefines the numbering style. Can't do it with titleformat command because
% the TOC doesn't display the right number style.
% This is the default numbering style.
\renewcommand{\thesection}{\thechapter.\arabic{section}}
\renewcommand{\thesubsection}{\thesection.\arabic{subsection}}
\renewcommand{\thesubsubsection}{\thesubsection.\arabic{subsubsection}}
