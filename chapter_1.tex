%%%%%%%%%%%%%%%%%%%%%%%%%%%%%%%%%%%%%%%%%%%%%%%%%%%%%%%%%%%
% Author: Sébastien Gardoll copyright all rights reserved %
% email: sebastien@gardoll.fr                             %
% LaTeX Project Public License 1.3c                       %
%%%%%%%%%%%%%%%%%%%%%%%%%%%%%%%%%%%%%%%%%%%%%%%%%%%%%%%%%%%
\chapter{Title of the chapter}
\label{chap:title_of_the_chapter}
Some preambule.

\section{Text emphasis}

\textbf{bold text}
\textit{italicied text}
\underline{underlined text}
\emph{emphasising text}

\section{Itemize example}

\begin{itemize}
\item {Item1: \lipsum[1]}
\item {Item2}
\item {Item3}
\end{itemize}
\begin{itemize}
\item {carriaged return item:}
\end{itemize}
\lipsum[1]

\section{Section}

\lipsum[1]
\lipsum[2]

\subsection{Subsection}

\lipsum[1]

\subsubsection{Subsubsection}

\lipsum[1-2]

\section{Image inclusions}

\ref{fig:pic1.png} is an example of a full width page inclusion
of a picture (aspect ratio is kept):

\includeimg{1}{pic1.png}{The caption of the picture.}

\ref{fig:pic2.png} is an example of a scaled down picture:

\includeimg{0.5}{pic2.png}{The caption of the picture.}

\ref{fig:pic3.png} is an example of a rotated picture:

\includeimgsideways{0.5}{pic3.png}{The rotated caption of the picture.}

\section{Citations}

According to \cite{inbook-full}...
This is a footnote citation\footfullcite{incollection-minimal}.

\section{Some fancy maths}

\[\textit{MSE}=E(Y-\hat{Y})^2=\underbrace{\left[F(X)-\hat{F}(X)\right]^2}_{\text{reductible}} + \underbrace{\textit{Var}(\varepsilon)}_{\text{irréductible}}\]

\[d=\left\{
        \begin{array}{l}
        1\ si\ A>\alpha \\
        0\ si\ A\leqslant\alpha
        \end{array}
        \right.
      \]

\[d(x_1,\ x_2)=\sqrt{\ \sum_{j=1}^n\frac{1}{f_n}(f_{1j}-f_{2j})^2}\]

\begin{theo}[Morgan's Law]
\[\overline{P \wedge Q} \Leftrightarrow \bar{P} \vee \bar{Q}\]
\[\overline{P \vee Q} \Leftrightarrow \bar{P} \wedge \bar{Q}\]
\end{theo}

\section{Tables}

From the CTABLE man (link \href{https://ctan.org/pkg/ctable}{here}):

\ctable[
cap     = The Skewing Angles,
caption = The Skewing Angles ($\beta$) for
$\fam0 Mu(H)+X_2$ and $\fam0 Mu(H)+HX$~\tmark,
label   = nowidth,
pos     = h
]{rlcc}{
\tnote{for the abstraction reaction,
$\fam0 Mu+HX \rightarrow MuH+X$.}
\tnote[b]{1 degree${} = \pi/180$ radians.}
\tnote[c]{this is a particularly long note, showing that
footnotes are set in raggedright mode as we don’t like
hyphenation in table footnotes.}
}{                                                          \FL
&            & $\fam0 H(Mu)+F_2$     & $\fam0 H(Mu)+Cl_2$ \ML
&$\beta$(H)  & $80.9^\circ$\tmark[b] & $83.2^\circ$       \NN
&$\beta$(Mu) & $86.7^\circ$          & $87.7^\circ$       \LL
}